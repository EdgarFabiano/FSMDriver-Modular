\section{Conclusion and Future Works}\label{sec:conclusions}
Digital games provide an excellent opportunity to experiment and study the 
different applications of Artificial Intelligence, such as controlling an autonomous
vehicle. The The Simulated Car Racing Championship is used as a standard test for 
such controllers, allowing different solutions compared in a straightforward manner.

This work presents the early stages of FSMDriver, a finite state machine approach
to developing a controller. The goal is to eventually compete in SCR by improving
the controller's behavior though machine learning techniques. This first version
considers four states for defining its racing behavior: \textbf{Straight Line}, 
\textbf{Curve}, \textbf{Out of Track}, and \textbf{Stuck}. Overall, the results
obtained in this world provided very valuable information to guide future 
developments of the proposal.

Initial experimental results showed that the approach is feasible, as FSMDriver
can successfully complete races and even be faster than some of the available
TORCS controllers. They also showed that adjusting configuration parameters lead 
to improved behavior, further efforts using machine learning techniques to 
improve this are being pursued.

Comparison to existing controllers showed that FDSMDriver's first version has 
average behavior, and there is much work to be done. To this end, future versions
will consider not only the parameters that define transitions between states, but
also the states' internal parameters.

Analysis of race conditions indicated an immediate need for a new state to handle 
between a straight stretch of track and a curve. Because experimental results 
showed extreme damage to the car due to crashing to the wall, it is likely that 
a state to \emph{avoid collisions} will be necessary. This could also be used to 
handle adversary cars.