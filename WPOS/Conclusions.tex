\section{Conclusion and Future Works}

	In this article it was presented a Finite State Machine approach, in its early days, to the problem of simulated car racing at the TORCS.
	
	At this stage the pilot showed can win just in few tracks and just of some bots available at the package TORCS. In aim to win some of the competitions at the conference it's necessary, at least, to be better than the bots on average of the various tracks, so there's a lot a work to do to achieve this goal. 
	
	It's also possible to run with the racers of the past simulated car racing, meaning that you know when you are ready to a real competition at the next conference.
	
	The creation of a state to deal with opponents it's the next step in effort to take a better performance at racing, without this it's necessary to count with some luck at the moment of interaction with some opponent.
	
	\toDo{Comentar sobre os resultados obtidos}
	
	The high damage received at the tests show us that it need some improvement. A close look show that at the moment of transition from a straight line to a curve the car lost the control and went out of track, and eventually, the car crashes somewhere and get a damage. One possible way to deal with this problem is an insertion of another state, that prepare the car to a curve, adjusting the speed and location above the axis of track. A filter on the brake can be a good 
	
	To summarize there's a vast open way to improvements, maybe using some evolutionary algorithm over the exhaustive search applied here to find a good combination of parameters.A change of paradigm also could be good, for instance, change the PID controller to some fuzzy logic. The advantage of a simulator is that all the questions about performance can be solved with the simulation.

