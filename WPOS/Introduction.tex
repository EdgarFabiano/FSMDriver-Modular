\section{Introduction}\label{sec:intro}
\IEEEPARstart{D}{igital} games provide a great test bed for experimentation and 
study of Artificial Intelligence (AI), and there has been a growing interest in 
applying AI in them, regardless of genre~\cite{simon2008}. Some applications 
include controlling non-playable characters~\cite{stanley2005}, path planning 
\cite{freitas2012}, human pose recognition~\cite{shotton2011}, and others. 

Electronic games also present a well defined environment, which may simulate 
extremely complex situations such as flying an airplane or controlling a car. One such simulator is \emph{The Open Racing Car Simulator} (TORCS)~\cite{TORCS}, whose input and output are constrained by the software's characteristics, providing an ideal benchmark for comparing AI solutions for creating car controllers.

Due to  the complexity of the task, most solutions attempt to develop an efficient
controller by evolving it completely, that is, by considering its performance on
the whole track. This is a very complex task, and in an attempt to
reduce this complexity, this works proposes the FSMDriver, a finite state machine 
based controller that divides the track into different types of stretches and 
evolves a state for handling each one.

The practical applications of such controllers in autonomous vehicles are enormous, 
and have been the subject of intense research for some years. For example, the 
DARPA\footnote{http://www.darpa.mil} has offered substantial prizes to winners of 
robotic challenges, the most famous being the Grand Challenge race won by Thrun 
et al.~\cite{Thrun06} .

This paper is organized as follows: Section \ref{sec:background} presents the details on TORCS and 
the Simulated Car Racing Championship and some background on finite state machines,
Section \ref{sec:fsm} describes the implementation of the FSMDriver, Section \ref{sec:results} presents 
initial experimental results, and Section \ref{sec:conclusions} provides concluding remarks.
