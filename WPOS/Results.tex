\section{Experimental Results}

The very first results showed that
% find an better expression than showed that
this controller achieves a better performance in more simple tracks, like oval ones, rather than in those with a higher degree of elaboration, like road or dirty tracks.

When racing in oval tracks, particularly in B-speedway, the FSMDriver outperforms some of the controllers provided by the TORCS distribution. The FSMDriver raced against considered average drivers and performed very well, winning even if it started behind them.

\begin{table}[h]
\renewcommand{\arraystretch}{1.3}
\caption{Results of a 3 Laps Race Held in B-Speedway against Berniw 10}
\label{table_1}
\centering
\begin{tabular}{c||c||c||c}
\hline
\bfseries Driver & \bfseries Total Time & \bfseries Damage & \bfseries Top Speed \\ 
\hline
\hline FSMDriver & 02:34:95 & 1 & 299 \\ 
\hline Berniw 10 & 02:40:59 & 8 & 282 \\ 
\hline 
\end{tabular}
\end{table}

\begin{table}[h]
\renewcommand{\arraystretch}{1.3}
\caption{Results of a 3 Laps Race Held in B-Speedway against Olethros 10}
\label{table_2}
\centering
\begin{tabular}{c||c||c||c}
\hline
\bfseries Driver & \bfseries Total Time & \bfseries Damage & \bfseries Top Speed \\ 
\hline
\hline FSMDriver & 02:34:88 & 1 & 299 \\ 
\hline Olethros 10 & 02:47:38 & 5 & 282 \\ 
\hline 
\end{tabular}
\end{table}

\begin{table}[h]
\renewcommand{\arraystretch}{1.3}
\caption{Results of a 3 Laps Race Held in B-Speedway against Iliaw 10}
\label{table_3}
\centering
\begin{tabular}{c||c||c||c}
\hline
\bfseries Driver & \bfseries Total Time & \bfseries Damage & \bfseries Top Speed \\ 
\hline
\hline FSMDriver & 02:34:90 & 1 & 299 \\ 
\hline Illaw 10 & 02:39:82 & 0 & 283 \\ 
\hline 
\end{tabular}
\end{table}



In the other hand, when facing considered good drivers the controller was defeated. However, if more than one driver race against FSMDriver at the same time, it will benefit from the crashes and finish in better positions.

\begin{table}[h]
\renewcommand{\arraystretch}{1.3}
\caption{Results of a 3 Laps Race Held in CG Speedway 1 against 6 considered good drivers}
\label{table_4}
\centering
\begin{tabular}{c||c||c||c}
\hline  \bfseries Driver & \bfseries Total Time & \bfseries  Damage & \bfseries Top Speed \\ 
\hline Inferno 3 & 02:30:92 & 0 & 300 \\ 
\hline Olethros 3 & 02:33:19 & 210 & 317 \\  
\hline bt 3 & 02:33:23 & 0 & 303 \\  
\hline FSMDriver & 02:33:30 & 315 & 309 \\  
\hline Berniw 3 & 02:39:52 & 0 & 312 \\
\hline Iliaw 3 & 02:40:62 & 1635 0 & 314 \\
\hline Tita 3 & 02:52:15 & 1794 & 300 \\
\hline 
\end{tabular}
\end{table} 


The damage taken can be significantly reduced by implementing an enemy avoidance and overtaking behavior,

Altough, when it comes to road tracks, especially the ones filled with critical curves combinations, the controller faces several problems at staying between the track boundaries. These troubles are mainly caused by entering curves with high speed values, fact that reinforces the idea of an approaching curve state, to deal with speed reductions and angle corrections. In addition, the constants used in transition process do not represent the optimal values for that role, an evolutionary method needs to be implemented in order to achieve those values. Moreover, oval tracks barely requires a state transition, in the overwhelming majority of time the car fits in the straight state. This lack of transitions and accentuated curves enabled
% is enable correct within this context?
a significantly gain in performance.

\subsection{Parameter changing}
Each state of the FSM uses specific parameters for ruling the state behavior. The transition function as well uses constants to decide whenever state must change. At the beginning those parameter were imported from the Simple Driver provided by Danielle Loiacono, as expected they did not lead to satisfying results when compared to the drivers available in TORCS.

For initial improvements the transition parameters have been chosen. This function mainly uses four constants:
\begin{itemize}
\item MAX\_SPEED\_DIST, the maximum distance for recognizing a curve.
\item LEFT\_EDGE, track position value that indicate the left boundary of the track.
\item RIGHT\_EDGE, track position value that indicate the right boundary of the track
\item STUCK\_TICKS, number of ticks required to leave the stuck state.
\end{itemize}

The exhaustive search was then applied, for each and every parameter an initial and maximum value were defined and at each iteration a delta parameter was added.

\begin{table}[h]
\renewcommand{\arraystretch}{1.3}
\caption{Parameters Used for Enhancement}
\label{parameter_table}
\centering
\begin{tabular}{c||c||c||c}
\hline \bfseries $p$ &\bfseries $p_0$ &\bfseries $ p_{max}$ &\bfseries $\Delta p$ \\
\hline
\hline MSD & $20$ & $200$ & $5$ \\ 
\hline LE & $-1$ & $-0.8$ & $0.1$ \\ 
\hline RE & $1$ & $0.8$ & $-0.1$ \\ 
\hline ST & $30$ & $300$ & $10$ \\ 
\hline 
\end{tabular} 
\end{table}

In the end 8658 combinations were generated and properly tested in \textit{CG Speedway 1}
for three laps. The best result generated used the following parameters: MAX\_SPEED\_DIST: $20$, LEFT\_EDGE: $-0.8$, RIGHT\_EDGE: $1$, STUCK\_TICKS: $110$.

\begin{table}[h]
\renewcommand{\arraystretch}{1.3}
\caption{Comparison Among Lupo Bianco, who achieved the best results in CG Speedway 1, Berniw 3 and The FSMDriver}
\label{results_table}
\centering

\begin{tabular}{c||c||c||c}
\hline \bfseries Driver &\bfseries Total Time &\bfseries Best Lap &\bfseries Top Speed \\
\hline
\hline Lupo Bianco & 01:50:12 & 00:35:22 & 262 \\
\hline Berniw 3 & 02:07:28 & 00:40:97 & 247 \\ 
\hline FSMDriver & 02:34:49 & 00:50:12 & 242 \\ 
\hline 
\end{tabular}
\end{table}
This approach demands a lot of time and space since all combinations are generated and executed. For future evolutions a more robust method, like genetic algorithm or neural networks needs to be applied in order to provide reliable results.