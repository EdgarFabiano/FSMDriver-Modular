\section{The Simulated Car Racing Championship}

The Simulated Car Racing Championship is \toDo{descrição mais completa ligação lógica com a subseção seguinte}. 

\subsection{The Open Car Racing Simulator}

TORCS is a platform focused on the establishment of ground rules towards the comparison of algorithms concerning \textbf{programming} and \textbf{AI}. The purpose is to create pilots to simulate real-life situations of racing competitions, and to do so, the Formula 1 score/awarding system is used as unbiased criteria.\\
The interface provides with a diversified set of sensors, such as the current position of the car in the track, its acceleration and brake values, and so on. Utilizing this sensors is the method with which the users communicate with their created pilots and the environment, and it is of their responsibility to program the entire behavior of this pilot, in whichever programming language of desire.\\
TORCS serves both as a research platform for AI on racing development and as an ordinary car racing game. The main reason why TORCS is widespread in the AI gaming universe is its portability; it runs on all \textit{Linux}'s architectures, \textit{FreeBSD}, \textit{OpenSolaris}, \textit{MacOSX} and both 32 and 64 bits \textit{Windows}. TORCS started some years ago to have its own competitions, where groups of development can submit their pilots in order to compete with others, and the goal of FSMDriver - the pilot of this publication's topic - is to be submitted to one of such competitions.

\subsection{Finite State Machines}
\toDo{Inserir teoria.}