\documentclass[journal]{IEEEtran}%
\usepackage[utf8]{inputenc}

\usepackage{graphicx}%
\usepackage[caption=false,font=footnotesize]{subfig}%
\usepackage{xcolor}%
\newcommand{\toDo}[1]{\textcolor{red}{#1}}%

\begin{document}
	\title{A Finite State Machine Controller for the Simulated Car Racing Championship}

\author{\IEEEauthorblockN{Bruno H. F. Macedo\IEEEauthorrefmark{1},
		Gabriel F. P. Araujo\IEEEauthorrefmark{1},
		Gabriel S. Silva\IEEEauthorrefmark{1},\\
		Matheus C. Crestani\IEEEauthorrefmark{1},
		Yuri B. Galli\IEEEauthorrefmark{1} and
		Guilherme N. Ramos\IEEEauthorrefmark{1}\IEEEauthorrefmark{2}\\[0.2cm]}
	\IEEEauthorblockA{\IEEEauthorrefmark{1}University of Brasília}
	\IEEEauthorblockA{\IEEEauthorrefmark{2}gnramos@unb.br}}

% The paper headers
%\markboth{Postgraduate Program Workshop in Computer Science}%
%{}%
%Provide better translation%
%\markboth{Workshop do Programa de P\'{o}s-Gradua\c{c}\~{a}o em Inform\'{a}tica - WPOS 2014}%
%{}%

% make the title area
\maketitle

% As a general rule, do not put math, special symbols or citations
% in the abstract or keywords.
\begin{abstract}
Electronic games can simulate extremely complex situations and be used for benchmark tests for specialized software. 
TORCS simulates a very realistic environment, and is ideal for comparing for autonomous car controllers and, consequently, the machine learning techniques applied in generating such controllers. This paper presents an AI approach to efficiently generating a car controller based on a finite state machine. Experimental results show how the controller's parameters are selected and their effect on its performance.
\end{abstract}

% Note that keywords are not normally used for peerreview papers.
\begin{IEEEkeywords}
finite state machines, computer games, Simulated Car Racing Championship, artificial intelligence
\end{IEEEkeywords}%

\section{Introduction}

\subsection{The TORCS environment}

\IEEEPARstart{T}{ORCS[1]} is a platform focused on the establishment of ground rules towards the comparison of algorithms concerning \textbf{programming} and \textbf{AI}. The purpose is to create pilots to simulate real-life situations of racing competitions, and to do so, the Formula 1 score/awarding system is used as unbiased criteria.\\
The interface provides with a diversified set of sensors, such as the current position of the car in the track, its acceleration and brake values, and so on. Utilizing this sensors is the method with which the users communicate with their created pilots and the environment, and it is of their responsibility to program the entire behavior of this pilot, in whichever programming language of desire.\\
TORCS serves both as a research platform for AI on racing development and as an ordinary car racing game. The main reason why TORCS is widespread in the AI gaming universe is its portability; it runs on all \textit{Linux}'s architectures, \textit{FreeBSD}, \textit{OpenSolaris}, \textit{MacOSX} and both 32 and 64 bits \textit{Windows}. TORCS started some years ago to have its own competitions, where groups of development can submit their pilots in order to compete with others, and the goal of FSMDriver - the pilot of this publication's topic - is to be submitted to one of such competitions.

\subsection{The FSMDriver}

The method of choice to submit the pilot to future TORCS competitions is, naturally, a \textbf{Finite State Machine}. A brief explanation of the behavior of this method will be given here, for more information[2], FSMs are widespread over the academic knowledge in books and in the web.
\textbf{Finite State Machines} (\textbf{FSM}), \textbf{Finite-State Automaton}, or even the simple \textbf{State Machine}, are designations of the same model. They all describe a \textbf{mathematical model of computation} used in computer programs, such as this one, and also in sequential logic circuits. FSM has a typical \textit{abstract machine} behavior, with a limited number of states, and, by definition, can be only and exclusively in one of these states at a time. The state in which the machine is in a present/defined moment will be hereby called the \textbf{current state}, as a matter of convenience. The machine changes its current state when triggered by a pre-defined event, also hereby called a \textbf{transition}.\\
Thus, to sum up, a FSM is a model defined by states, which the \textit{current state} is changed every time a \textit{transition} defined by the designated conditions is triggered.\\[0.2cm]
A race inside TORCS has some peculiarities perhaps only visible for those who possess a deeper understanding on the behavior of a real car, however, a layperson can absolutely develop a pilot in the environment, given the easy features the interface provides. The communication between programmer and car is the main aspect that led the choice of a FSM for the pilot. FSMs have been one of the favorite instruments for AI gaming, and they are present in almost every game due to many reasons as follows:

\begin{itemize}
	\item \textbf{Readiness}: A FSM is quickly and simply implemented, in almost every one of its forms;
	
	\item \textbf{Modularity}: Debugging and reworking a FSM become catalyzed processes due to the abstraction of its structure, that is, its behavior can be divided into smaller, independent parts, which can be treated separately in a way that reduces labor;
	
	\item \textbf{Low computational complexity}: A FSM would hardly ever require great amounts of processor time, and that is because its operation mainly follows but hard-coded rules;
	
	\item \textbf{Intuitive behavior}: Analyzing a FSM is an easy process because human minds are often used to categorizing situations and conditions of conduct. This characteristic is both a good employment for debugging in real time and for turning the pilot's behavior simple enough so that a common person, i.e. non-programmer, could identify mistakes and spectate a race understanding the development of the pilot;
	
	\item \textbf{Flexibility}: A fundamental quirk of a FSM, one which makes the process of expanding the scope of the structure easier in a level hardly other choice of method would provide. If a new situation occurs, one that was not foreseen by the states defined and also one that could not be inserted in any of them, the mere creation of a new state would solve the matter, in about no time and almost without interference in the other parts of the whole.
\end{itemize}

The division of states for the FSMDriver itself are in agreement with the advantages of a FSM, meaning that the choice of which states to be used relied on the human intuition of the development of a car in operation. The states designed are:

\begin{itemize}
	\item \textbf{Straight line}: The state in which the pilot commands the car to full throttle and maximum speed/acceleration ahead, and occurs when the car is located in a situation that sees no turning nearby;
	
	\item \textbf{Curve}: Situations focused on the steering of the car, that is, changing its direction of movement in order to keep it inside the delimitations of the proper racing space and also to avoid crashing the car into an obstacle;
	
	\item \textbf{Out of track}: When the previous states fail to prevent the car from entering an undesired situation such as going outside of the track, where conditions such as friction are much worse for the car than inside the track, this state takes control of the car and tries to return it to the race;
	
	\item \textbf{Stuck}: In a worst case scenario, when the car is barely moving, drastic measures need to be taken, such as reversing the car and steering it out of an obstacle. The appearence of such events reduce largely the performance of the pilot in the race.
\end{itemize}

In other words, the pilot will encounter four situations in a race, two of them are desired, whereas the other two are emergency actions. While the car is either on a \textbf{straight line} or on a \textbf{curve}, the behavior is normal; but when the criteria defined to manipulate the pilot into entering those states fails to keep it racing, and the car goes \textbf{outside of the track} or gets \textbf{stuck}, the car encounters troubles and loses performance. That is why the accurate definition of the requirements the pilot has to fulfill to be in each of these states needs to be thought thoroughly for good and competitive results.
%
\section{Background}\label{sec:background}

The Simulated Car Racing Championship (SCR) is a well-known event comprising 
three sequential competitions held in association with \textit{IEEE}, and 
present in well known conferences such as the \textit{Congress on Evolutionary 
Computation} (CEC), \textit{ACM Genetic and Evolutionary Computational Conference} 
(GECCO) and the \textit{Symposium on Computational Intelligence and Games} 
(CIG)~\cite{scr2009}.  It provides an opportunity for the scientific community 
to perform a straightforward comparison among different approaches in complex 
environments containing multiple continuous variables~\cite{caldeira_2013}. For 
this, it uses \emph{The Open Car Racing Simulator}, a state-of-the art car simulator. 

\subsection{The Open Car Racing Simulator}
TORCS provides a very customizable environment, sophisticated 3D graphics, and -
most importantly - a powerful physics simulation platform~\cite{manual}. Its 
physics engine considers static and dynamic aspects, such as fuel usage, damage 
received, wheel traction, and others, in a very detailed way. Thus, it enables 
a very complex environment for testing AI techniques.

TORCS serves both as a research platform for AI on racing development and as an 
ordinary car racing game. The main reason why TORCS is widespread in the AI gaming 
universe is its portability; it runs on all \textit{Linux}'s architectures,
\textit{FreeBSD}, \textit{OpenSolaris}, \textit{MacOSX} and both 32 and 64 bits 
\textit{Windows}. TORCS has its own online community, which started its own 
racing competitions some years ago~\cite{TORCS}, where developers can submit 
their controllers to compete with each other.The simulator's version used at the present work is torcs 1.3.4.

\subsection{The Simulated Car Racing Championship}
The SCR contest provides a standard measure for TORCS controllers by defining strict
rules, based on the Formula 1 score system~\cite{scr2009}. It improves on the 
practical applications aspects by limiting the controller's knowledge of 
the system to the information provided by its inputs and its actions by its outputs,
much like an actual autonomous vehicle.

The interface provides with a diversified set of sensors, such as the current 
position of the car in the track, its acceleration and brake values, and so on~\cite{scr2009}. 
The information provided by them defines the controllers perception of its environment (track, other controllers, etc.), and the idea is to program a behavior that results
in the best possible race. 

\subsection{SCR Contestants}
Works submitted to SCR have a wide range of variety, going from sophisticated 
heuristic designs to completely mathematical and statistical approaches. The 
controller that won the 2009 Simulated Car Racing Championship,~\cite{scr2009}
perhaps one of the most important editions realized so far, was created by 
Onieva et al.~\cite{onieva_2009}, and it 
consists of a simple set of controllers in an \emph{modular configuration}. 

Although this controller won the competition, the authors noted that it evolved 
into a fairly complex model and noted that a modular approach considerably reduce 
the effort to implement a  controller. Refactoring and debugging such a complex 
architecture requires changing code that is not directly related to the desired
change in behavior.

Another interesting controller used  artificial neural networks~\cite{cardamone_2010},
a more general approach which required little domain knowledge and provided a 
satisfactory result. Also in this case,  the implementation is quite complex and 
the final result manages connections between all inputs and outputs.

In order to reduce the effort in developing a controller, a more modular approach,
such as a \emph{finite state machine}, is desirable.

\subsection{Finite State Machines}\label{sec:fsm}

A \emph{Finite State Machines} (FSM) designates the model of a device, which has 
a finite number of states it can be in at a given moment, and which and can 
operate on input to either transition from one state to another or to cause an 
output to take place~\cite{buckland2005}. The machine can only be in one of these 
states at a given time.

FSMs have been widely used in AI applications in games~\cite{millington_2009}, due 
its inherent characteristics:

\begin{itemize}
	\item \textbf{Simplicity}: this model can be quickly implemented in many ways.
	
	\item \textbf{Flexibility}: new states can be easily incorporated into the model.
	
	\item \textbf{Modularity}: states can be independently developed in parallel.
	
	\item \textbf{Focused debugging}: since FSMs can only be in one state at a time,
	debugging/testing is limited to the state in question.
	
	\item \textbf{Small computational overhead}: states are usually implemented 
	with hard-coded rules concerning a specific situation, which in turn demands
	small amounts of processor time.
	
	\item \textbf{Intuitive behavior}: analyzing a FSM is an easy process for 
	humans because we are always intuitively categorizing situations and behaviors,
	just like a FSM working. 
\end{itemize}
%
\documentclass{UnBeamer}%

% Codificação
\usepackage[brazilian]{babel}%
\usepackage[utf8]{inputenc}%

% Inicialização
\title{A Finite State Machine Controller for the Simulated Car Racing Championship}%
\subtitle{WPOS 2014}%
\author[gnramos]{Guilherme N. Ramos}%
\institute[CIC]{Departamento de Ciência da Computação\\Universidade de Brasília}%
\date[2014/1]{}%

\graphicspath{{img/}}%

\begin{document}%
	\frame{\maketitle}% 
\end{document}%%
\section{Experimental Results}
%
\section{Conclusion and Future Works}

	In this article it was presented a Finite State Machine approach, in its early days, to the problem of simulated car racing at the TORCS.
	
	At this stage the pilot showed can win just in few tracks and just of some bots available at the package TORCS. The code developed aim to win some of the competitions at the conference it's necessary, at least, to be better than the bots on average of the various tracks, so there's a lot a work to do to achieve this goal. 
	
	It's also possible to run with the racers of the past simulated car racing, meaning that you know when you are ready to a real competition at the next conference.
	
	The creation of a state to deal with opponents it's the next step in effort to take a better performance at racing, without this it's necessary to count with some luck at the moment of interaction with some opponent.
	
	\toDo{Comentar sobre os resultados obtidos}
	
	The high damage received at the tests show us that it need some improvement. A close look show that at the moment of transition from a straight line to a curve the car lost the control and went out of track, and eventually, the car crashes somewhere and get a damage. One possible way to deal with this problem is an insertion of another state, that prepare the car to a curve, adjusting the speed and location above the axis of track. A filter on the brake can be a good solution in order to minimaze the speed without allowing the car to skid on the track.   
	
	To summarize there's a vast open way to improvements, maybe using some evolutionary algorithm over the exhaustive search applied here to find a good combination of parameters.A change of paradigm also could be good, for instance, change the PID controller to some fuzzy logic. The advantage of a simulator is that all the questions about performance can be solved with the simulation.

%

\bibliographystyle{IEEEtran}%
\bibliography{bibliografia}%

\end{document}
